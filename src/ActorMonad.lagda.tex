\begin{code}
module ActorMonad where
open import Membership using (_∈_ ; _⊆_ ; xs⊆xs)

open import Data.List using (List ; [] ; _∷_)
open import Data.Unit using (⊤ ; tt)

open import Coinduction using (∞ ; ♯_)
open import Level using (Lift ; lift ; suc ; zero)
\end{code}

An Actor is indexed by the shape of its inbox.

The shape dictates what types the messages sent to an actor can have,
 and thus also what types the messages being received by the actor can have.

The shape is constant over the actors whole life-time.
It would be nice to allow the shape to grow monotonically over time.
The use of coinduction is a product of wanting to allow actors to have mutual references
 Alternative solutions are welcome

Values and references are kept separate,
since the sending and receiving of values have different behaviour from sending and receiving references.

%<*InboxShape>
Our typing rules for communication are based on the idea that the most important property of a message is that it can be understood by the receiver.
The type system will thus be used to limit what type a message sent to an inbox can have.
By limiting the types of messages sent to an inbox,
we can make sure that every message read from the inbox has the expected type.
In terms of cite!43 years!, this is the interface of the actor.

There are two kinds of messages: values and references.
A value is any data from Agda's lowest set universe.
Typical examples are \AgdaRef[Examples/Types]{Bool}, \AgdaRef[Examples/Types]{ℕ}, or \AgdaRef[Examples/Types]{⊤}.
We limit the types to the lowest set universe since it's not clear how values of higher universes could be serialized
and to be sent over the wire in a distributed setting, since that would entail serializing the types themselves.
Unfortunately, the lowest set universe also contain funcions that have their types fully specified:
\AgdaCatch{Examples/Types}{unfortunate}
We could of course add further constraints on the types, e.g. that every value has to be serializable,
but due to its insignificance to the calculus we have opted not to.

Compared to values, a reference type is a bit more involved.

\AgdaTarget{InboxShape}
\AgdaTarget{ValueType}
\AgdaTarget{ReferenceTypes}
\begin{code}
mutual
  record InboxShape : Set₁ where
    inductive
    constructor ⊠[V:_][R:_]
    field
      value-types : ValueTypes
      reference-types : ReferenceTypes

  ReferenceTypes = List InboxShape
  ValueTypes = List Set
\end{code}

\todo{Reference the type pollution problem}
\todo{List of values, list of inbox types}
\todo{Better than Either?}

The \AgdaRef[ActorMonad]{InboxShape} is a variant type that details all the messages that can be received from an inbox.

We can think of the \AgdaRef[ActorMonad]{InboxShape} as a set of types,
and every received message comes paired with a proof that the message is a type from that set.

To know what type the message has, the receiver simply has to inspect the proof,
Agda will unify the constraints, and the value or reference will be accessible.

We create two convenient properties expressing that a type is a value or a reference in the mentioned set:
\begin{code}
_is-value-in_ : Set → InboxShape → Set₁
V is-value-in S = V ∈ InboxShape.value-types S

_is-reference-in_ : InboxShape → InboxShape → Set₁
R is-reference-in S = R ∈ InboxShape.reference-types S
\end{code}
%</InboxShape>

%<*ReferenceSubtype>
As mentioned, the important property of a message is that it can be understood by the receiver.
We could also say that the type of the message has to be compatible with the specified type.
This property is of course a subtype relation,
which states that a type is related to another type by some notion of substitutability.

Let's focus on what the subtype relation should be for a message containing a reference.
When an actor receives a reference, the only thing the reference can be used for is to send messages over.
What the sending actor wants, is that any message sent to the reference has a valid type for that inbox.

\begin{code}

-- An inbox shape can be used in place of another if
-- it accepts every value and reference of the other.
record [_]-handles-all-of-[_] (actual wanted : InboxShape) : Set₁ where
  field
    values-sub : InboxShape.value-types wanted ⊆ InboxShape.value-types actual
    references-sub : InboxShape.reference-types wanted ⊆ InboxShape.reference-types actual

-- A reference can be used in place of another reference in S if
-- it accepts every value and reference of the other,
-- and of course, the other has to be in S
record [_]-is-super-reference-in-[_] (Fw S : InboxShape) : Set₁ where
  field
    {wanted} : InboxShape
    wanted-is-reference : wanted is-reference-in S
    fw-handles-wanted : [ Fw ]-handles-all-of-[ wanted ]

\end{code}
%</ReferenceSubtype>

\begin{code}

-- We can create a value message for an inbox of shape S,
-- if the type of the value is a value type for S.
data ValueMessage (S : InboxShape) : Set₁ where
  Value : ∀ {A} → A is-value-in S → A → ValueMessage S

-- We can create a reference message for an inbox of shape S,
-- if the type of the reference is a reference type for S.
--
-- When a reference message is received, the actors capabilities will increase,
-- allowing the actor to send messages to the actor referenced by the message.
--
-- A reference message can be created without having a valid reference in the current context.
-- It is the constructors of ActorM (more specifically SendReference) that limits the sending
-- of references to only those that are valid in the current context.
--
-- We index ReferenceMessage by both the reference type and the receiver's inbox.
data ReferenceMessage (S Fw : InboxShape) : Set₁ where
  Reference : [ Fw ]-is-super-reference-in-[ S ] → ReferenceMessage S Fw

-- A Message is either a value or a reference.
--
-- We could just have wrapped ValueMessage and ReferenceMessage,
-- but that makes for a noisier experience when pattern matching in application code.
data Message (S : InboxShape): Set₁ where
  Value : ∀ {A} → A is-value-in S → A → Message S
  Reference : ∀ {Fw} → Fw is-reference-in S → Message S

-- Simple lifting of ⊤ to reduce noise when the monad returns ⊤
⊤₁ : Set₁
⊤₁ = Lift ⊤

-- When a message is received, we increase our capabilities iff the message is a reference.
add-if-reference : ∀ {S} → ReferenceTypes → Message S → ReferenceTypes
add-if-reference pre (Value _ _) = pre
add-if-reference pre (Reference {Fw} _) = Fw ∷ pre

infixl 1 _>>=_

-- An Actor is modeled as a monad.
--
-- It is indexed by the shape of its inbox, which can't change over the course of its life-time.
--
--
-- 'A' is the return value of the monad.
--
-- 'pre' is the precondition on the list of references that are available.
-- The precondition is what limits an actor to only being able to send messages
-- to actors that it has a reference to.
-- Sending a message is done by indexing into 'es',
-- thereby proving that the actor has a reference to the actor
-- that it's sending the message to.
--
-- 'post' is the postcondition on the list of references.
-- The postcondition sometimes depends on what happens during runtime,
-- and is thus modelled as a function on 'A'.
-- 'post' is what enables receive to have the right type.
data ActorM (IS : InboxShape) : (A : Set₁) →
              (pre : ReferenceTypes) →
              (post : A → ReferenceTypes) →
              Set₂ where
  -- Value is also known as return.
  -- The precondition is the same as the assignment axiom schema in Hoare logic.
  Value : ∀ {A post} → (val : A) → ActorM IS A (post val) post
  -- Bind / composition
  -- This is the same as the rule of composition in Hoare logic.
  -- post₁ is the midcondition.
  _>>=_ : ∀ {A B pre post₁ post₂} → (m : ∞ (ActorM IS A pre post₁)) →
          (f : (x : A) → ∞ (ActorM IS B (post₁ x) (post₂))) →
          ActorM IS B pre post₂
  -- Spawn a new actor.
  -- The spawned actor does not know any references.
  -- The reference to the spawned actor is added to the parent actor.
  Spawn : {NewIS : InboxShape} → {A : Set₁} → ∀ {pre postN} →
          ActorM NewIS A [] postN →
          ActorM IS ⊤₁ pre λ _ → NewIS ∷ pre
  -- Send a value to an actor.
  -- A value can only be sent to an actor if a reference to it is
  -- available in the precondition.
  -- ValueMessage is indexed by the shape of the inbox we're sending to,
  -- which makes sure that it's not possible to send values of the wrong type.
  SendValue : ∀ {pre} → {ToIS : InboxShape} →
    (canSendTo : ToIS ∈ pre) →
    (msg : ValueMessage ToIS) →
    ActorM IS ⊤₁ pre (λ _ → pre)
  -- Send a reference to an actor.
  -- A reference can only be sent to an actor if a reference to it is
  -- available in the precondition.
  -- The reference being sent also has to be available in the precondition.
  -- ReferenceMessage is indexed by the shape of both the shape of the forwarded
  -- reference and the shape of the receiving inbox.
  SendReference : ∀ {pre} → {ToIS FwIS : InboxShape} →
    (canSendTo : ToIS ∈ pre) →
    (canForward : FwIS ∈ pre) →
    (msg : ReferenceMessage ToIS FwIS) →
    ActorM IS ⊤₁ pre (λ _ → pre)
  -- Receive a message.
  -- When receiving a message, the postcondition depends on whether the message
  -- is a value or a reference.
  -- If the message is a value, the postcondition is the same as the precondition.
  -- If the message is a reference, the postcondition is the reference cons'ed to the precondition.
  -- If a receive is encountered when there are no messages in the actor's inbox,
  -- then the actor is moved to the 'blocked queue'.
  -- Sending a message to a blocked actor will move the actor from the 'blocked queue' back to the
  -- active actors.
  Receive : ∀ {pre} → ActorM IS (Message IS) pre (add-if-reference pre)
  -- Lift let's you call a sub-program that needs less references than what is currently available.
  -- To allow that a lifted program increases the available references,
  -- the postcondition of the resulting actor is the same as the postcondition of the lifted program.¨
  -- We'd like there to be a way of re-adding the forgotten references, but that's easy to implement.
  -- To implement re-adding references we'd have to carry around what references to re-add when the
  -- lifted part is finished.
  ALift   : ∀ {A ys post xs} → (inc : ys ⊆ xs) →
    ∞ (ActorM IS A ys post) →
    ActorM IS A xs post
  -- Adds the reference to this actor to its available references.
  Self : ∀ {pre} → ActorM IS ⊤₁ pre (λ _ → IS ∷ pre)

--
-- ========== Helpers for ActorM ==========
--

-- coinduction helper for Value
return₁ : {A : Set (suc zero)} → ∀ {IS post} → (val : A) → ∞ (ActorM IS A (post val) post)
return₁ val = ♯ Value val

-- universe lifting for return₁
return : {A : Set} → ∀ {IS post} → (val : A) → ∞ (ActorM IS (Lift A) (post (lift val)) post)
return val = return₁ (lift val)

-- coinduction helper for spawn
spawn : ∀ {IS NewIS A pre postN} →
  ActorM NewIS A [] postN →
  ∞ (ActorM IS ⊤₁ pre λ _ → NewIS ∷ pre)
spawn newAct = ♯ Spawn newAct

-- coinduction helper and neater syntax for value sending
_!v_ : ∀ {IS ToIS pre} →
  (canSendTo : ToIS ∈ pre) →
  (msg : ValueMessage ToIS) →
  ∞ (ActorM IS ⊤₁ pre (λ _ → pre))
canSendTo !v msg = ♯ SendValue canSendTo msg

-- coinduction helper and neater syntax for reference sending
to_!r_via_ : ∀ {IS pre} → {ToIS FwIS : InboxShape} →
  (canSendTo : ToIS ∈ pre) →
  (msg : ReferenceMessage ToIS FwIS)
  (canForward : FwIS ∈ pre) →
  ∞ (ActorM IS ⊤₁ pre (λ _ → pre))
to canSendTo !r msg via canForward = ♯ SendReference canSendTo canForward msg

-- coinduction helper for Receive
receive : ∀ {IS pre} → ∞ (ActorM IS (Message IS) pre (add-if-reference pre))
receive = ♯ Receive

-- An inbox can handle every value and reference of itself
handles-self : {IS : InboxShape} → [ IS ]-handles-all-of-[ IS ]
handles-self = record { values-sub = xs⊆xs ; references-sub = xs⊆xs }

⊠-of-values : ValueTypes → InboxShape
⊠-of-values vals = ⊠[V: vals ][R: [] ]
\end{code}
